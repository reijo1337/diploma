\chapter{Экспериментальный раздел}
\label{cha:research}

В рамках дипломного проекта был ряд экспериментов, направленных на исследование построенного метода распознавания жестовых интерфейсов. Целью проведенных исследований является выяснение зависимости качества классификации от этапа предобработки и количества итераций в алгоритме динамической маршрутизации.

В качестве входных данных использовались изображения жестов, выполненные как мужчинами, так и женщинами различной национальности.

\section{Описание тестовых данных}

Для проведения вычислительных экспериментов использовались следующие наборы данных:

\begin{itemize}
	\item ASL Finger Spelling Dataset -- набор изображений дактилей американского жестового языка. Использовался для оценки качества распознавания в описанных ранее методах \cite{Karn,Starner,Garcia}. На основании результатов данной выборке делается вывод о качестве построенного метода относительно конкурентов. Данный датасет состоит из двух частей: изображений 24 дактилических жестов (в данную выборку не входят буквы <<j>> и <<z>>, так как являются динамическими) и карт глубин. В данной работе использовалась первая часть, которая состоит из 65000 изображений с непостоянным размером в цветовом пространстве RGB. Жесты демонстрируются пятью разными людьми.
	\item RSL by Oleg Potkin -- набор данных русского дактиля. Содержит 1042 RGB изображения размером 128$\times$128 пикселей. Разделен на 10 классов-букв: <<а>>, <<б>>, <<в>>, <<г>>, <<е>>, <<и>>, <<о>>, <<п>>, <<с>>.
	\item Numbers -- набор данных c изображением жестов цифр. Состоит из из 1125 RGB изображений.
\end{itemize}

\section{Формальная модель и описание условий исследования}

Для выявления зависимости качества распознавания от количества итераций в алгоритме динамической маршрутизации в рамках исследования для одного набора данных строились модели для двух, трех, четырех, пяти, шести и семи итераций. Каждая модель обучалась на предобработанных и оригинальных наборах данных.

Для проведения вычислительных экспериментов была разработана формальная модель, представленная на рисунке \ref{res:research}.

\begin{figure}
	\centering
	\includegraphics[width=\textwidth]{inc/img/research}
	\caption{Формальная модель эксперимента}
	\label{res:research}
\end{figure}

В качестве метрики качества распознавания метода используется вероятность корректной классификации жестового символа (формула \ref{res:acc}).

\begin{equation}
\label{res:acc}
\text{Точность} = \frac{\text{Верные классификации}}{\text{Ложные классификации} + \text{Верные классификации}}
\end{equation}

Каждый набор изображений был разделен в соотношении 20\% для валидации, 64\% для обучения и 16\% для тестирования. 

Исследования проводились с использованием платформы Google Colaboratory, предоставляющая бесплатное выполнение файлов ipython notebook с использованием GPU и TPU. В рамках одной сессии предоставляется 25,51 Гб ОЗУ и 68,40 дискового пространства.

\section{Результаты исследований}

Для оптимальной настройки описанного метода были проведены вычислительные эксперименты с целью определения зависимости точности распознавания от числа итераций динамичесой маршрутизации. Для выяснения влияния этапа предобработки на качество работы классификатора данные вычисления проводились для обработанных и исходных наборов изображений. Результаты экспериментов были обобщены и виде графиков, представленных на рисунках \ref{res:asl}, \ref{res:rsl_oleg} и \ref{res:number}

\begin{figure}[!h]
	\centering
	\includegraphics[width=0.9\textwidth]{inc/img/asl}
	\caption{Зависимость точности распознавания от предобработки входных данных и числе итераций на датасете ASL Finger Spelling Dataset}
	\label{res:asl}
\end{figure}

\begin{figure}[!h]
	\centering
	\includegraphics[width=0.9\textwidth]{inc/img/rsl_oleg}
	\caption{Зависимость точности распознавания от предобработки входных данных и числе итераций на датасете RSL by Oleg Potkin}
	\label{res:rsl_oleg}
\end{figure}

\begin{figure}[!h]
	\centering
	\includegraphics[width=0.9\textwidth]{inc/img/rsl_hse}
	\caption{Зависимость точности распознавания от предобработки входных данных и числе итераций на датасете Numbers}
	\label{res:number}
\end{figure}

Исследование показало, что предобработка изображений позволяет увеличить точность распознавания в среднем на 10-15 \%, как видно на рисунках \ref{res:asl} и \ref{res:rsl_oleg}. С другой стороны, есть вероятность получения зашумленных предобработанных изображений, следствием чего является потеря работоспособности классификатора (рисунок \ref{res:number}). Пример зашумленного предобработанного изображения представлен на рисунке \ref{res:bad_preproc}.

\begin{figure}[ht!]
	\begin{minipage}[h]{0.49\linewidth}
		\center{\includegraphics[width=0.5\linewidth]{inc/img/preprocessing/bad_orig} \\ а)}
	\end{minipage}
	\hfill
	\begin{minipage}[h]{0.49\linewidth}
		\center{\includegraphics[width=0.5\linewidth]{inc/img/preprocessing/bad_res} \\ б)}
	\end{minipage}
	\caption{Результат предобработки изображения из датасета Number с зашумлением: а) исходное изображение; б) предобработанное изображение}
	\label{res:bad_preproc}
\end{figure}

Наибольшая точность распознавания достигается при трех итерациях алгоритма маршрутизации, как видно на рисунках \ref{res:asl} и \ref{res:rsl_oleg}.

\section{Анализ полученных результатов}

В связи с тем, что набор ASL  Finger Spelling Dataset использовался в ряде дру­гих исследований в области распознавания жестовых символов, опуб­ликованных в последние годы, стало возможным сравнить результаты распознавания предложенного метода с аналогичными методами.

Точность распознавания для других методов представлена на ри­сункe \ref{res:compare}.

\begin{figure}[!h]
	\centering
	\includegraphics[width=\textwidth]{inc/img/compare}
	\caption{Сравнение точности распознавания известных методов}
	\label{res:compare}
\end{figure}

Полученные результаты показывают, что разработанный метод на 3\% позволяет повысить точность распознавания жестовых символов.
