\Defines % Необходимые определения. Вряд ли понадобться
В настоящей работе используются следующие термины с соответствующими определениями.
\begin{description}
	\item [Жест] осознанное движение человеческого тела или его части, которое несет информацию и совершается с целью её передачи.
%\item[Дискретизация] преобразование непрерывного информационного множества аналоговых сигналов в дискретное множество.
%\item[Музыкальное произведение] последовательность сочетаний звуков.
%\item[Звук] физическое явление, представляющее собой распространение в виде упругих волн механических колебаний в твёрдой, жидкой или газообразной среде.
%\item[Колебания] повторяющийся в той или иной степени во времени процесс изменения состояний системы около точки равновесия. Колебания характеризуются частотой, фазой и амплитудой.
%\item[Частота] физическая величина, характеристика периодического процесса, равна количеству повторений или возникновения событий в единицу времени. Единица измерения -- герцы (Гц).
%\item[Фаза] аргумент периодической функции, описывающей колебательный или волновой процесс.
%\item[Амплитуда] максимальное значение смещения или изменения перемен­ ной величины от среднего значения при колебательном или волновом движении.
%Применительно к музыке вместо общего термина <<звук>> используют более конкретный термин <<музыкальный звук>>.
%\item[Музыкальный звук] это звук определённой высоты, использующийся
%как материал для создания музыкальных сочинений. Для письменной фиксации му­зыкальных звуков используется нотация. Наиболее распространённые формы записи высотных значений музыкальных звуков – латинская буквенная (C, D, E, F, G, A, B/H) или слоговая (до, ре, ми, фа, соль, ля, си) нотация. В данной работе для обозначения музыкальных звуков используется латинская буквенная нотация с указанием номера октавы.
%\item[Нота] музыкальный звук или его письменная запись.

%Далее понятие <<звук>> используется в смысле музыкального звука.

%\item[Аккомпанемент] сопровождение одним или несколькими инструментами, а также оркестром сольной партии (певца, инструменталиста, хора и других). Сопроводителя называют аккомпаниатором. Аккомпанементом также называют гармоническое и ритмическое сопровождение основной мелодии, голоса.
%\item[Вокал (или пение)] исполнение мелодии с помощью голоса человека.
%\item[Тембр] (обертоновая) окраска звука, <<качество тона>>.
%\item[Рефакторинг] процесс изменения внутренней структуры программы, не затрагивающий её внешнего поведения и имеющий целью облегчить понимание её работы.
%\item[Цифровая звуковая рабочая станция] электронная или компьютерная система, предназначенная для записи, хранения, редактирования и воспроизведения цифрового звука. Предусматривает возможность выполнения на ней законченного цикла работ, от первичной записи до получения готового результата.
\end{description}

%%% Local Variables:
%%% mode: latex
%%% TeX-master: "rpz"
%%% End:
