% Также можно использовать \Referat, как в оригинале
\Referat
%\begin{abstract}
Дипломная работа: 64 страницы, 24 рисунка, 8 таблиц, 30 источников.

В магистерской диссертации рассмотрена задача распознавания статических жестов. Проведен анализ предметной области, проанализированы известные методы решения данной задачи и обоснована необходимость построения метода, объединяющего преимущества существующих подходов. Разработан метод распознавания, основанны на использовании искусственных нейронных сетей, модифицированный путем добавления предобработки изображений с целью упрощения процесса классификации и использования в качестве классификатора капсульный нейронных сетей. Разработаны алгоритмические структуры для реализации комбинированного метода. Осуществлена программная реализация предложенного решения. Прове­дены вычислительные исследования, подтверждающие работоспособ­ность метода и показывающие точность распознавания свыше 95\%.

Результаты исследовательской деятельности были отражены в печатной работе.

В первом разделе приводится сравнительный анализ известных методов распознавания жестовых символов, на основании которого выбран прототип, для которого будут проводится исследования. Обоснована необходимость построение нового метода. Во втором разделе описан метод распознавания жестовых символов, описаны структуры, необходимые для его реализации. В третьем разделе описана архитектура программной реализации предложенного метода, обоснован выбор средств программной реализации. Описаны ресурсы, необходимые для сборки и запуска разработанного программного обеспечения, форматы входных и выходных файлов, а также интерфейс и руководство пользователя. В четвертом разделе описаны и проведены вычислительные экспери­ менты для исследования предложенного метода.

%\end{abstract}

%%% Local Variables: 
%%% mode: latex
%%% TeX-master: "rpz"
%%% End: 
