% Также можно использовать \Referat, как в оригинале
\Referat
%\begin{abstract}
РПЗ 67 страниц, 20 рисунков, 11 таблиц, 21 источник.

Объектом исследования данной работы являются музыкальные произведения.
Целью рабоы является разработка метода автоматизированного выделения голосовой составляющей из музыкальной композиции.

Задачи:

\begin{itemize}
	\item анализ предметной области;
	\item анализ методов выделения вокальной партии из музыкальных композиций;
	\item разработка собственного метода на основании проанализированных;
	\item разработка программного комплекса, реализующего выбранный метод;
	\item анализ эффективности работы программного продукта.
\end{itemize}

Способы применения программного продукта: программный продукт может применяться для дальнейшей разработки выделения компонентов сведенного аудио сигнала, выделения голосовой составляющей из песен для автоматизированного создания текста, выделения голосовой составляющей из аудио дорожки видеофайла для улучшения автоматизированной генерации субтитров. 
%\end{abstract}

%%% Local Variables: 
%%% mode: latex
%%% TeX-master: "rpz"
%%% End: 
