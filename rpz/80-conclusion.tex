\Conclusion % заключение к отчёту

В результате проделанной работы были решены следующие задачи:

\begin{itemize}
	\item был проведен анализ предметной области;
	\item были проанализированы существующие решения;
	\item на основе полученных во время анализа данных был разработан собственный метод выделения голосовой составляющей из монофонического аудио сигнала;
	\item предложенный метод был реализован в программном продукте.
\end{itemize}

В результате тестирования и эксперимента было установлено, что разработанный метод:

\begin{itemize}
	\item для вокала имеет примерно те же показатели метрик, что и метод FASST, при этом имея в среднем в два раза меньший разброс, для аккомпанемента же средние значения метрик в среднем на 60\% лучше, имея примерно те же значения разброса;
	\item для вокала значение метрик в среднем на 80\% лучше, чем у метода ГНС, при этом дисперсия в среднем в два раза меньше. Для аккомпанемента значение средних метрик приблизительно одинаков, но значение дисперсии у разработанного метода на 30\% меньше;
	\item для вокала значение метрик в среднем в 2,5 раза уступают методу СНС, при этом дисперсия в среднем в 2 раза лучше. Для аккомпанемента метрики примерно одинаковые, но разброс у разработанного метода в среднем в 2 раза больше, чем у метода СНС.
\end{itemize}

В результате тестирования и эксплуатации разработанного ПО заменен основной недостаток -- наличие примесей из соседних источников в выделяемом сигнале

Развитие разработанного метода можно осуществлять по следующим направлениям:

\begin{itemize}
	\item увеличение качества выделения переработкой архитектуры сети;
	\item определение источников, участвовавших в записи исходного сигнала.
\end{itemize}

%%% Local Variables: 
%%% mode: latex
%%% TeX-master: "rpz"
%%% End: 
