\Conclusion % заключение к отчёту

В результате проделанной работы были решены следующие задачи:

\begin{itemize}
	\item был проведен анализ предметной области;
	\item были проанализированы существующие решения;
	\item на основе полученных во время анализа данных был разработан собственный метода распознавания жетовых символов;
	\item предложенный метод был реализован в программном продукте;
	\item был проведен анализ качества разработанного метода.
\end{itemize}

В результате тестирования и эксперимента было установлено, что разработанный метод имеет зависимость от качества изображения, полученного на этапе обработки: повышение точности при нормальном изображении и потеря работоспособности при зашумленном. Так же показано, что данный метод позволяет увеличить точность распознавания на 3\% в сравнении с известными аналогами.

Результаты исследовательской деятельности были отражены в печатной работе\cite{Tantsevov}.

Развитие разработанного метода можно осуществлять по следующим направлениям:

\begin{itemize}
	\item модификация цветового фильтра путем автоматизации подбора конфигурации;
	\item увеличение скорости обучения и работы, а также качества работы классификатора путем оптимизации предложенной архитектуры, добавлением сверточных и капсульных слоев, применением другого алгоритма динамической маршрутизации.
\end{itemize}

%%% Local Variables: 
%%% mode: latex
%%% TeX-master: "rpz"
%%% End: 
