\Introduction

Одной из актуальных задач в области компьютерного зрения является распознавание жестов. Качественный метод распознавания жестов позволит дать развитие многим системам, таким как интеллектуальные жестовые интерфейсы, системы перевода с жестовых языков, управление для систем виртуальной и дополненной реальностей. В настоящее время известен ряд практически применимых решений данной задачи \cite{Suharjito}, но все они имеют недостатки, например необходимость использования дополнительных источников данных (перчатки с датчиками).

Целью даной работы является разработка метода распознавания жестовых символов. Для решения поставленной задачи необходимо:

\begin{itemize}
	\item проанализировать предметную область;
	\item проанализировать существующие методы распознавания жестов;
	\item на основе полученных во время анализа данных разработать собственный метод распознавания жестовых символов;
	\item реализовать разработанный метод в программном продукте;
	\item провести планирование и постановку экспериментов с целью выяснения качества работы разработанного метода.
\end{itemize}