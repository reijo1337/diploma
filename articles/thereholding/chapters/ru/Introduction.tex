\section{Введение}
\label{sec:Intro}

Последние года активное развитие получили задачи компьютерного зрения. Одной из актуальных тем этой области является распознавание человеческих образов, в частности жестов. Качественный метод распознавания жестов позволит дать развитие многим системам, например жестовые интерфейсы, системы перевода с жестовых языков, управление для систем виртуальной и дополненной реальностей и так далее. На данный момент существует множество решений данной задачи\cite{slr}, но все они имеют недостатки, например необходимость использования дополнительных источников данных (перчатки с датчиками). 
Как правило, такие системы состоят из трех частей:

\begin{enumerate}
	\item Получение данных о жесте
	\item Предобработка данных
	\item Классификация
\end{enumerate}

Скорость и качество работы алгоритмов классификации во многом зависит от исходных данных. Например, для классификации жестов с помощью скрытой марковской модели \cite{inproceedings} основные признаки получаются из изображения рук в разноцветных перчатках. Тем самым, подобрать метод предобработки изображения таким образом, чтобы его применение в итоговом методе упрощало процесс классификации, не увеличивая при этом общее время работы. Алгоритмы, применимые для достижения данной цели, можно разделить на следующие группы:
\begin{itemize}
	\item Выделение контура фигуры
	\item Выделение силуэта кисти руки
	\item Построение скелета кисти руки
\end{itemize}

Далее рассмотрим каждую из этих групп.