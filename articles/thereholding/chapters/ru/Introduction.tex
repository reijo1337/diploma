\section{Введение}
\label{sec:Intro}

Одной из перспективных задач машинного зрения является распознавание жестовых символов. Актуальность данной темы обусловлена множеством сфер применения: от новых методов взаимодействия с ПК до систем распознавания жестовых языков. 
Как правило, такие системы состоят из трех частей:

\begin{enumerate}
	\item Получение данных о жесте
	\item Предобработка данных
	\item Классификация
\end{enumerate}

В качестве исходных данных можно использовать снимок с камеры, например, смартфона. В этом случае, для упрощения классификации, важен этап предобработки данных. На данном этапе необходимо выделить на изображении основные признаки исходного жеста. Алгоритмы, применимые для достижения данной цели, можно разделить на следующие группы:
\begin{itemize}
	\item Выделение контура фигуры
	\item Выделение силуэта кисти руки
	\item Построение скелета кисти руки
\end{itemize}

