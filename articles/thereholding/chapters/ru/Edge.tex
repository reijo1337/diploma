\section{Выделение контура фигуры}
\label{sec:Edge}

Для выделения контура кисти руки можно использовать операторы преобразования изображения. К таким методам можно отнести:

\begin{itemize}
	\item Оператор Собеля
	\item Оператор Прюитт
	\item Перекрестный оператор Робертса
	\item Оператор Кэнни
\end{itemize}

Рассмотрим каждый подробнее:

\subsection{Оператор Собеля}

Основная идея оператора Собеля \cite{Sobel} заключается в вычислении градиента освещенности каждой точки изображения. Вычисление производится примерное с помощью свертки изображения двумя сепарабельными целочисленными фильтрами размера 3x3 в вертикальном и горизонтальном направлениях. Благодаря этому вычисление работа данного оператора имеет низкие трудозатраты. В результате получаются два новых изображения Gx и Gy, в каждой точке которого записано приближенное значение производных по x и по y соответственно. Пусть A - исходное изображение, тогда вычисляются они следующим образом:

\begin{eqnarray}\label{eq:sobel-matrixs}
G_x = \begin{bmatrix}
-1 & 0 & 1\\
-2 & 0 & 2\\
-1 & 0 & 1\\
\end{bmatrix} \\
G_y = \begin{bmatrix}
-1 & -2 & -1\\
0 & 0 & 0\\
1 & 2 & 1\\
\end{bmatrix}
\end{eqnarray}

Определение данных матриц на языке Python выглядит следующим образом:

\begin{lstlisting}
sobelx = [[-1, 0, 1], [-2, 0, 2], [-1, 0, 1]]
sobely = [[1, 2, 1], [0, 0, 0], [-1, -2, -1]]
\end{lstlisting}

В итоге значение градиента вычисляется как $G=\sqrt{G_x^2+G_y^2}$, а его направление как $\theta=\arctan(\frac{G_x}{G_y})$.

Результат показывает скорость изменения яркости изображения в конкретной точке, т.е. вероятность ее нахождения на границе изображения.

\begin{lstlisting}
for row in range(self.width-len(sobelx)):
for col in range(self.height-len(sobelx)):
gx = 0
gy = 0
for i in range(len(sobelx)):
for j in range(len(sobely)):
val = mat[row+i, col+j] * lin_scale
gx += sobelx[i][j] * val
gy += sobely[i][j] * val

pixels[row+1, col+1] = int(math.sqrt(gx*gx + gy*gy))
\end{lstlisting}