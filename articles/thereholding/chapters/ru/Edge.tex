\section{Выделение контура фигуры}
\label{sec:Edge}

Для выделения контура кисти руки можно использовать детекторы границ, основная идея которых заключается в поиске градиента изменения яркости изображения. Они работают исключительно с черно-белыми. То есть нулевым шагом данных методов можно указать преобразование изображения из цветного в черно-белое. К таким методам можно отнести:

\begin{itemize}
	\item Оператор Собеля \cite{Sobel}.
	\item Оператор Прюитт \cite{Prewitt}.
	\item Перекрестный оператор Робертса \cite{Roberts}.
	\item Оператор Кэнни \cite{Canny}.
\end{itemize}

Рассмотрим каждый метод подробнее:

\subsection{Оператор Собеля}

Основная идея оператора Собеля\cite{Sobel} заключается в вычислении градиента освещенности каждой точки изображения. Работа данного метода заключается в  свертке изображения двумя сепарабельными целочисленными фильтрами размера 3x3 в вертикальном и горизонтальном направлениях (рисунок \ref{fig:sobel}). Благодаря этому вычислительная работа данного оператора имеет низкие трудозатраты. Схема алгоритма представлена на рисунке \ref{fig:sobel-block}.

\begin{figure*}[!h]
	\centering
	\includegraphics[width=0.5\textwidth,keepaspectratio]{figures/ru/sobel-block.png}
	\caption{Общая схема алгоритмов определения границ изображения}
	\label{fig:sobel-block}
\end{figure*}

Ядра свертки в данной схеме определяются следующим образом:

\begin{eqnarray}\label{eq:sobel-matrixs}
G_x = \begin{bmatrix}
-1 & 0 & 1\\
-2 & 0 & 2\\
-1 & 0 & 1\\
\end{bmatrix} \\
G_y = \begin{bmatrix}
-1 & -2 & -1\\
0 & 0 & 0\\
1 & 2 & 1\\
\end{bmatrix}
\end{eqnarray}

\begin{figure*}[!h]
	\centering
	\includegraphics[width=\textwidth,keepaspectratio]{figures/ru/sobel.png}
	\caption{Свертка изображения для получения контура}
	\label{fig:sobel}
\end{figure*}

Значение градиента вычисляется как $G=\sqrt{G_x^2+G_y^2}$, а его направление как $\theta=\arctan(\frac{G_x}{G_y})$.

Результат показывает скорость изменения яркости изображения в конкретной точке, т.е. вероятность нахождения в данном месте границы изображения.

\subsection{Оператор Прюитт}

Принцип работы данного алгоритма\cite{Prewitt}, как и оператор Собеля, соответствует схеме на рисунке \ref{fig:sobel-block}. Отличие заключается в способе задачи ядра:

\begin{eqnarray}\label{eq:prewitt-matrixs}
G_x = \begin{bmatrix}
-1 & 0 & 1\\
-1 & 0 & 1\\
-1 & 0 & 1\\
\end{bmatrix} \\
G_y = \begin{bmatrix}
-1 & -1 & -1\\
0 & 0 & 0\\
1 & 1 & 1\\
\end{bmatrix}
\end{eqnarray}

Из-за меньшего значения средних элементов итоговое изображение имеет более явный эффект сглаживания.

\subsection{Перекрестный оператор Робертса}

Рассмотрим область 3х3, представленную на рисунке \ref{fig:roberst}.

\begin{figure*}[!h]
	\centering
	\includegraphics[width=0.2\textwidth,keepaspectratio]{figures/ru/roberts}
	\caption{Окрестность 3х3 внутри изображения}
	\label{fig:roberst}
\end{figure*}

Вычисление первых частных производных, которые обозначают перепад яркости, в центральной точке $z_5$ можно провести следующим образом:

\begin{eqnarray}\label{eq:roberts-eq}
G_x = (z_9 - z_5) \\
G_y = (z_8 - z_6)
\end{eqnarray}

Для вычисления данных производных в каждой точке изображения в данном методе\cite{Roberts} применяется свертка изображения двумя ядрами размера 2х2:

\begin{eqnarray}\label{eq:roberts-matrixs}
\begin{bmatrix}
1 & 0\\
0 & -1
\end{bmatrix} 
\begin{bmatrix}
0 & 1\\
-1 & 0
\end{bmatrix}
\end{eqnarray}

Тем самым данный алгоритм полностью соответствует схеме \ref{fig:sobel-block}.

В результате получается изображение пространственного градиента исходного изображения, где точки с наибольшим значением соответствуют границе.

Проблемой данного метода является отсутствие четко выраженного центрального элемента у ядра свертки. Но в следствии этого недостатка алгоритм так же имеет высокую скорость обработки изображения.

\subsection{Оператор Кэнни}

Данный фильтр\cite{Canny} был разработан с учетом удовлетворения следующим условиям:
\begin{itemize}
	\item хорошее обнаружение (Кэнни трактовал это свойство как повышение отношения сигнал/шум);
	\item хорошая локализация (правильное определение положения границы);
	\item единственный отклик на одну границу.
\end{itemize}

Схема  работы алгоритма представлена на рисунке \ref{fig:canny_block}. Рассмотрим каждый этап подробнее с наглядной визуализацией обработки. Для этого применим данный оператор шаг за шагом к изображению \ref{fig:canny} a.
\begin{figure*}[!h]
	\centering
	\includegraphics[width=0.2\textwidth,keepaspectratio]{figures/ru/canny-block}
	\caption{Блок-схема алгоритма работы оператора Кэнни}
	\label{fig:canny_block}
\end{figure*}

\begin{enumerate}
	\item Размытие изображения для удаления лишнего шума. Для этого можно применить фильтр Гаусса\cite{shapiro:2001}.
	Функция размытия, используемая этим фильтром, для двумерного случая задается формулой  \ref{eq:gauss}.
	\begin{eqnarray}\label{eq:gauss}
	Gaus(x, y, \sigma) = \frac{1}{2 \pi \sigma^2}*e^{\frac{-(x^2+y^2)}{2\sigma^2}}
	\end{eqnarray}
	
	Результат применения фильтра Гаусса к изображению \ref{fig:canny}a представлен на рисунке \ref{fig:canny}б.
	
	\item Поиск градиентов, для определения границ с максимальным значением градиента.
	На данном этапе можно использовать оператор Собеля \cite{Sobel}, работа которого была описана выше.
	
	Результат поиска градиентов в размытом изображении представлен на рисунке \ref{fig:canny}в.
	
	\item Подавление не-максимумов, т.е. исключение из границ не локальных максимумов.
	
	На данном шаге происходит проверка, является ли конкретный пиксель локальным максимумом вдоль направления градиента. Таким образом исключаются ложные границы. Результат экспериментального исследования данного этапа представлен на рисунке \ref{fig:canny}г.
	
	\item Определение потенциальных границ с помощью двойной пороговой фильтрации.
	
	Фильтр использует два порога фильтрации:
	\begin{itemize}
		\item Все пиксели со значением больше верхней границы принимают максимальное значение (достоверная граница).
		\item Все пиксели со значением меньше нижней границы подавляются.
		\item Все пиксели со значением в диапазоне границ принимают фиксированное среднее значение. Их уточнение происходит на следующем этапе.
	\end{itemize}

	Пример фильтрации с порогами 0,01 и 0,07 представлен на рисунке \ref{fig:canny}д.

	\item Трассировка области неоднозначности
	
	На данном этапе происходит разделение пикселей, получивших промежуточное значение на предыдущем шаге, на границы и фон (увеличение значения и подавление). Пиксель добавляется к границе, если он соприкасается с ней по одному из 8-ми направлений.
	
	Результат работы оператора Кэнни представлен на рисунке \ref{fig:canny}е.
\end{enumerate}


\begin{figure}[ht!]
	\begin{minipage}[h]{0.49\linewidth}
		\center{\includegraphics[width=0.5\linewidth]{figures/ru/bmstu_gray} \\ а)}
	\end{minipage}
	\hfill
	\begin{minipage}[h]{0.49\linewidth}
		\center{\includegraphics[width=0.5\linewidth]{figures/ru/bmstu_smoothed} \\ б)}
	\end{minipage}
	\begin{minipage}[h]{0.49\linewidth}
		\center{\includegraphics[width=0.5\linewidth]{figures/ru/bmstu_gradient} \\ в)}
	\end{minipage}
	\hfill
	\begin{minipage}[h]{0.49\linewidth}
		\center{\includegraphics[width=0.5\linewidth]{figures/ru/bmstu_non_max} \\ г)}
	\end{minipage}
\begin{minipage}[h]{0.49\linewidth}
\center{\includegraphics[width=0.5\linewidth]{figures/ru/bmstu_threshold} \\ д)}
\end{minipage}
\hfill
\begin{minipage}[h]{0.49\linewidth}
\center{\includegraphics[width=0.5\linewidth]{figures/ru/bmstu_final} \\ е)}
\end{minipage}
	\caption{Результаты экспериментального исследования этапов оператора Кэнни: а) исходное изображение; б) применение размытия; в) поиск градиентов; г) подавление не-максимумов; д) двойная пороговая фильтрация; г) трассировка областей неоднозначности.}
	\label{fig:canny}
\end{figure}