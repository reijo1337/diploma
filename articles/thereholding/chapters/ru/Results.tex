\section{Анализ результатов экспериментальных исследований}
\label{sec:Results}

Целью экспериментов было определение наиболее оптимального метода преобразования исходного изображения с целью упрощения процесса классификации. Для этого необходимо сравнить визуально результаты работы перечисленных выше алгоритмов и время их работы. Результаты экспериментов представлены ниже, отдельно для каждого набора данных:

\begin{itemize}
	\item Результаты для ASL Alphabet представлены на рисунке \ref{fig:asl} и в таблице \ref{tab:asl-alphaber}
	\item Результаты для Hand Gesture of the Colombian sign language представлены на рисунке \ref{fig:colombian} и в таблице \ref{tab:colombian-alphaber}
	\item Результаты для ASL Fingerspelling Images представлены на рисунке \ref{fig:asl2} и в таблице \ref{tab:asl2-alphaber}
	\item Результаты для sign language between 0 9 представлены на рисунке \ref{fig:datamix} и в таблице \ref{tab:datamix-alphaber}
\end{itemize}



В результате экспериментов установлено, что среди методов выделения контура по качеству работы лидирует оператор Кэнни. Учитывая небольшую разницу во времени их работы, можно отбросить из рассмотрения все остальные операторы.

Простое выделение силуэта показало наилучшие временные результаты. Так же при правильной предварительной настройке метода можно добиться удовлетворительной четкости выделения. Тем не менее, предварительная настройка является главной проблемой этого алгоритма.

Морфологическое построение скелета показало плохой результат. Как говорилось выше, в результате получаются побочные ветви, а так же скелет получается неполносвязным. Данные недостатки не позволят сильно упросить работу классификатора в силу зашумленности итоговых данных.

Алгоритм построения скелета по ключевым точкам не справился со своей задачей на большинстве результатов. Так же для любого типа данных он работает за одно и тоже время. Это одновременно и хорошо (результаты Hand Gesture of the Colombian sign language) и плохо (остальные результаты).