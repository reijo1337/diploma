\section{Построение скелета кисти руки}
\label{sec:Skeleton}

Для ускорения процесса классификации жеста руки можно использовать скелетную модель. Данный тип входных данных в силу своей специфики может упростить вычисление признаков, необходимых классификатору.

Для построения скелета кисти можно использовать метод построения скелета выпуклой фигуры \cite{DIP}. 

В качестве выпуклой фигуры можно использовать результат работы метода, описанного в разделе \ref{sec:Threholding}.

ТУТ НУЖНО ОПИСАНИЕ МЕТОДА

Проблемой данного метода являются побочные ветви скелета, образованные из-за возможной зашумленности или неточности фигуры. Другим недостатком можно считать отсутствие гарантии обеспечения связного набора пикселей для всего скелета, или обеспечения одинаковой ширины ветвей во всем скелете.

