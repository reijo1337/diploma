\section{Заключение}
\label{sec:Conclusion}

В результате данной работы были исследованы возможные методы предобработки изображения в задачах классификации жестовых символов. Были проведены эксперименты с целью определения наиболее оптимального по скорости работы и качеству выделения основных признаков метода. 

В результате сравнительного анализа можно выделить два метода:

\begin{itemize}
	\item Выделение силуэта. Данный метод показал нработыаименьшее время работы, Кроме того, бинарное изображение руки содержит в себе необходимые признаки жеста для его обработки классификатором.
	\item Построение скелета по ключевым точкам. Скелетная модель является наиллучшим типом входных данныз для классификатора, т.к. не несет в себе никаких лишних данных\cite{10.1007/978-3-642-31298-4_16}.
\end{itemize}

Первый метод показал наилучшие результаты по скорости работы алгоритма, кроме тестов на широкоформатных изображениях. В дальнейшем можно предложить улучшение путем упрощения первичной конфигурации цвета кожи и оптимизации работы на больших изображениях.

Второй метод, несмотря на неудачные результаты тестов, расходящимся с результатами авторов\cite{DNN}, можно попробовать оптимизировать по качеству через переобучение модели. По скорости данный алгоритм можно оптимизировать путем реализации его на другом языке программирования.