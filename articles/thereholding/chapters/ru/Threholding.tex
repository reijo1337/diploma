\section{Выделение силуэта кисти руки}
\label{sec:Threholding}

Помимо классических методов определения границ можно использовать сегментацию по цвету кожи \cite{Phung}. Данный метод преобразует RGB изображение в бинарное с помощью фильтрации пикселей по цвету, близкому к цвету кожи. Для улучшения работы алгоритма перед фильтрацией изображение переводят в цветовое пространство YCrCb, в котором различные цвета кожи  расположены близко друг к другу \cite{Siddharth}.

Пример фильтрации изображения: 

\begin{minipage}{0.75\textwidth}
	\begin{algorithm}[H]
		\lstinputlisting[language=Python]{src/color_filter.py}
		\caption{Фильтрация изображения по цвету кожи}
		\label{imp:color-filter}
	\end{algorithm}
\end{minipage}

В большинстве случаев после бинаризации на изображении присутствуют шумы и артефакты, вызванные тем, что на фоновой части изображения находились пиксели, попадающие в ограничения фильтра. Для их устранения можно использовать морфологические операции "расширение" и "сужение" \cite{DIP}:

\begin{minipage}{0.75\textwidth}
	\begin{algorithm}[H]
		\lstinputlisting[language=Python]{src/color_noise.py}
		\caption{Применение к изображению операций расширение и сужение}
		\label{imp:color-noise}
	\end{algorithm}
\end{minipage}