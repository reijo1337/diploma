\section{Выделение силуэта кисти руки}
\label{sec:Threholding}

Помимо классических методов определения границ можно использовать сегментацию по цвету кожи \cite{Phung}. Данный метод преобразует RGB изображение в бинарное с помощью фильтрации пикселей по цвету, близкому к цвету кожи. Для улучшения работы алгоритма перед фильтрацией изображение переводят в цветовое пространство YCrCb, в котором различные цвета кожи  расположены близко друг к другу \cite{Siddharth}.

Данный метод предусматривает обработку каждого пикселя независимо от других, проверяя его на принадлежность заданному диапазону. Это позволяет ускорить работу алгоритма реализацией параллельной обработки отдельных пикселей.

В большинстве случаев после бинаризации на изображении присутствуют шумы и артефакты, вызванные тем, что на фоновой части изображения находились пиксели, попадающие в ограничения фильтра. Для их устранения можно использовать морфологические операции: "наращивание" и "эрозия" \cite{DIP}:

\begin{itemize}
	\item Наращивание. Задается как $A \oplus B = \cup_{b \in B}A_b$. 
	\item Эрозия. 
\end{itemize}