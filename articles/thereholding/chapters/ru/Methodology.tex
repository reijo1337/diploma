\section{Методология}
\label{sec:Method}

Для сравнительного анализа описанных выше методов они были реализованы на языке Python 3 с использованием следующих библиотек:

\begin{itemize}
	\item Python Imaging Library
	\item scipy
	\item numpy
	\item OpenCV
\end{itemize} 

Каждый алгоритмом был обработан одинаковый набор данных, состоящий из растровых изображений кистей рук тестовая выборка составлялась из нескольких наборов данных, различающиеся разными форматами изображений, их размерами и людьми, чьи руки использовались для получения изображений кистей. Все данные находятся в открытом доступе:

\begin{itemize}
	\item ASL Alphabet. Image data set for alphabets in the American Sign Language\cite{AslAlphabet}. Данный датасет состоит из 87000 изображений американского дактиля в обучающей выборке и 29 проверочных изображений. Каждое изображение цветное, сохранено в формате JPG, имеет размерность 200x200 пикселей. Для проверки работы алгоритмов были использованы проверочные изображения.
	
	\item Hand Gesture of the Colombian sign language. Hand gestures, recognizing the numbers from 0 to 5 and the vowels\cite{Colombian}. Данный датасет состоит из фотографий жестов, изображающий гласные буквы колумбийского языка и цифры от 0 до 5. Приведены снимки как мужских, так и женских рук. Каждый жест имеет 3 разных фото с разных ракурсов. Каждое изображение цветное, сохранено в формате JPG, имеет размерность 4608 x 2592 пикселей. Для проверки работы алгоритмов были отобраны случайные изображения, по одному на каждый жест.
	
	\item ASL Fingerspelling Images (RGB \& Depth) \cite{asl2}. Данная выборка состоит из изображений американского дактиля. В выборке участвуют 5 разных людей, у каждого человека на каждый символ приходится более 1000 изображений. Все изображения цветные, формата PNG, имеют различную размерность. Для проверки работы алгоритмов были отобраны случайным образом по одному изображению для каждой буквы, то есть в итоговой выборке участвовали снимки разных людей под разными ракурсами.
	
	\item sign language between 0 9\cite{sl09}. Данная выборка состоит из изображений жестов, обозначающих цифры от 0 до 10. Все изображения цветные, формата JPG, размерности 300x300. Данные разделены на обучающие, где на каждую цифру приходится более 100 различных изображений, и проверочные, где на каждую цифру представлено одно изображение. Для проверки работы алгоритмов были выбраны все изображения из проверочной выборки. 
	
\end{itemize}

 Были проведены замеры времени обработки каждого изображения для получения статистики по минимальному, максимальному и среднему времени работы алгоритма.
 
 Эксперимент проводился на ноутбуке Lenovo ThinkPad E580 со следующими конфигурациями:
 
 \begin{itemize}
 	\item Операционная система Linux Mint 19 Cinnamon
 	\item Четырех ядерный процессор Intel© Core™ i5 с тактовой частотой 1.60 ГГц
 	\item 8 Гб оперативной памяти
 \end{itemize}